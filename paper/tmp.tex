
\section{Political ambition and democracy}

Contemporary legislative studies generate key hypotheses from Mayhew's \citeyearpar{mayhew.1974} model of lawmakers. The iconic work on the electoral connection of members of the U.S.\ Congress views legislators as automatons with a unique, all-encompassing goal: reelection. The crucial premise in the argument is motivational: only one spring moves the incumbent, the ambition to stay another term in office. Mayhew does not deny that other worries might leave incumbents sleepless---turning some prioritary program into policy, climbing the chamber's hierarchy, her historical legacy are just some examples. But none could be achieved if the incumbent failed at her attempt to reelect. Despite its parsimony, the model explains most activity that representatives engage in while in Congress. 

Another premise is instrumental: reelection is a function of the incumbent's reputation among constituents. In personalistic systems, such as the U.S., reputation is mostly individual---to such degree that Mayhew discards the possibility that heterogeneous American parties could be of theoretical interest \citep[but revisionists rescued party relevance in Mayhew's framework,][]{cox.mccubbins.2007leviath2nd,aldrich.1995}. The instrumental premise merits three comments. 

First, it does not involve every constituent in the district but a subset. Groups making reelection much harder if they dropped their support for the incumbent are more important than others. \citet{cox.mccubbins.1986} call them \emph{core constituents}. From this perspective, it is rational and easier to work in preserving a coalition that made you win than attempting to build a new one.

Second, coalition maintenance requires delivering results, channeling new benefits to the core while preserving existing ones. As in all human relations, perceptions matter as much as substance: the core must give the incumbent credit for delivering. 

With collective production goods, where each member's effort is not immediately evident, the allocation of responsibility is far from automatic. Success has many parents. Thus the importance of particularistic goods, in contrast to more universalistic ones. Their distinguishing trait is that their production and/or delivery depends on the incumbent's personal effort \citep{haggard.mccubbins.2001}. Classic examples are from \citet{cain.etal.1987}: constituency service (service responsiveness) and pork-barrel legislation, construction spending earmarked for the district (allocation responsiveness). Incumbents have full control to direct pork where the political logic indicates, creating (this is crucial) a responsibility link.

To the extent that Mayhew's logic intersects with ascription problems, theory expects legislators to devote substantial time to cultivate their personal vote through delivery of particularistic goods. As a result, a closer link develops between lawmaker and her core constituents than the rest of the citizenry. As a consequence, better incumbent's name recognition is expected in the district than beyond. 



This approach to redistricting offers leverage to separate and measure campaign and incumbency effects. The instrument is name familiarity:

building a personal vote requires visibility, measured by name familiarity. 


.

Name recognition can stem from constituency service costrued broadly, in the form of both casework to help constituents in dealing with government agencies and allocation responsiveness or pork for the district. These are paticularistic benefits that the member delivers while in office and therefore can credibly claim credit for.

There is survey evidence that, other things constant, respondents who met the incumbent personally, or heard her/him speak, or even talked to staff were twice as likely in the U.S. and 1.5 times as likely in the U.K. to correctly recall their name \citep[][:34]{cain.etal.1987}.  

Two forces must be distinguished as they operate jointly on name recognition: incumbency (discussed above) and campaign effects.

different effects in incumbents' name recognition among constituents, summarized in Table \ref{T:hyps}. Two forces must be distinguished as they operate jointly on name recognition: incumbency (discussed above) and campaign effects.

The bulk of effort in a legislative campaign takes place in the district \citep{langston.nd}. Billboards and wall paintings, printed flier distribution and robocalls, meetings with neighbors alone or in the company of candidates higher in the ticket, or even vote-buying with construction material and debit cards are some examples of focalized effort. The effect in the candidate's name recognition occurs \emph{throughout the district} (i.e., the son). In contrast, the effect of incumbency in name recognition occurs in the area that father and son share. This generates our first predictions.

The probability that a voter picked at random among constituents (in lands $r$ or $w$) recognizes the candidate's name is substantially higher than a voter picked at random from outside the district (in $l$ or in $n$). By itself, the campaign effect generates no difference in recognition between areas $r$ and $i$. Neither does it among $l$ and $n$. The seconf column in Table \ref{T:hyps} reports these predictions. They can be summarized as $r=w>l=n$. 




 

https://portalanterior.ine.mx/archivos3/portal/historico/contenido/interiores/Menu_Principal-id-Mesas_Distritaciones_Electorales/
