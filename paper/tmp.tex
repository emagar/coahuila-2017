
\noindent We investigate candidate name recognition in races for the state of Coahuila assembly in 2017. Name familiarity has been associated with efforts by therepresentatives to cultivate a personal vote towards reeelection. We exploit redistricting prior to the races to identify differentials in name familiarity attributable theoretically to incumbency effects---and not to campaign effects, which occur simultaneously. Even if the instrument failed to include sufficient sampling points for a full separation due to few incumbents on the ballot, we detect significant shifts in name recognition in accordance with theoretical expectations. Survey evidence of the first election held after Mexico recently dropped single-term limits suggests that the few ambitious lawmakers solidified their electoral connection. 




Estudiamos el reconocimiento del nombre de candidatos al Congreso del estado de Coahuila en 2017. El fenómeno ha sido asociado con el esfuerzo del representante en su distrito para preservar su reelegibilidad. Aprovechamos la redistritación del estado que antecedió a la elección para detectar diferencias en reconocimiento atribuibles al efecto del ocupante y no al efecto de campaña. Aunque la cobertura muestral de la encuesta preeelectoral que usamos impide una separación cabal de los dos efectos, detectamos diferenciales en reconocimiento de nombre significativos y consistentes con la teoría. Ofrecemos tres diseños de investigación alternativos para que futuros estudios de opinión separen el efecto de ocupante (\emph{incumbency effect}) en elecciones que permitirán la reelección consecutiva a partir de 2018 en México. 

----


It is unfortunate that the survey exclude sampling points in area $g$ that districts gained. This precludes separation. Only predictions in Table \ref{T:Hyps}'s lines 1 and 4 are at odds, and both involve comparisons to land gained. The survey remains theoretically interesting to verify empirically total effects in lines 2, 3, and 5. 

My 2 cents: a dictator, in the Roman constitution, was meant to be a *temporary* one-man rule to confront a crisis. Autocrats perpetuate their rule. But often used interchangeably  


