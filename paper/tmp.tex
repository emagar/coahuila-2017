
cornerstone
dv = name recognition --- personal vote
reelection, coahuila
survey, moto-magar
incumbency --- simult to campaign
redistricting

\section{Introduction}

Is reelection the cornerstone of responsive government? Much political theory holds this dear. It is a tenet of what \citet[][:9]{riker.lib.pop.1978} calls the liberal or Madisonian democratic ideal: ``the function of voting is to control officials, and no more.'' In Schumpeter's \citeyearpar{schumpeter.1942} classic account, liberal democracy is a procedure giving voters a periodic opportunity of accepting or refusing those who are to rule them. Anticipation of voter retribution keeps rulers in check. 

%``Democracy means only that the people have the opportunity of accepting or refusing the men who are to rule them'' (285)

We investigate voters' familiarity with state assembly representatives in the state of Coahuila, in northern Mexico. Name familiarity is the standard approach to measure what \citet[][:9]{cain.etal.1987} call the personal vote, ``that portion of a candidate's electoral support which originates in his or her personal qualities, qualifications, activities, and record.'' Recent removal of single-term limits in Mexico offers a unique opportunity to investigate institutional change \citep[see][]{magarInstReel.2017}. Prior to the reform, incumbents across the board were constitutionally forced to retire. The manuscript joins the few investigations of consecutive reelection in Mexico. Such studies focus on plenary speech. \citet{motolinia-reel-pork2021} uncovers a substantial inter-term surge in plenary time devoted to particularistic legislation in post-reform state assemblies relative to the rest. In the federal Congress, \citet{magar.debate.2021} finds that single-member district deputies freed of single-term limits made significantly more and longer speeches than the rest, even after controlling for other correlates such as the member's party size and majority status, seniority, and the position in the chamber hierarchy. We contribute by offering a first view of the reform from the perspective of public opinion.

Campaigns are another short-term force shaping district electoral outcomes \citep{moreno.decisElec.2009,downs.1957,jacobson.1990spending,popkin.1991}. Simultaneous campaigns raise an obstacle to gauge the importance of the personal vote---unless the seat is open, which removes the incumbency effect, challengers campaign to unseat an incumbent. We propose a research design that exploits redistricting to separate these simultaneous effects. Periodic changes in district boundary delimitation, which also took place in Coahuila prior to state assembly races we study, migrate some groups from one district to another. So even with incumbents running for another term in office, these voters will not find theirs' on the ballot. 

to conc: The procedure is applicable to other systems promoting the personal vote \citep{carey.shugart.1995} where districts are re-drawn periodically.

%Another short-term force is the effect of campaigns \citep{moreno.decisElec.2009,downs.1957,jacobson.1990spending}. Campaign effects are successful attempts to shift a prior coalition, by breaking it or by expanding it towards new groups and interests. ``Campaigns transform unstructured and diverse interests into a single coalition, making a single cleavage dominant'' \citep[][:12]{popkin.1991}. 

We added items to tap attitudes towards reelection---including name recognition \citep[cf.][]{cain.etal.1987}---among respondents to a pre-election survey in Coahuila's state races in 2017, the first post-reform ballot. Identifying precincts that mapmakers moved across districts, we test expectations of differentials in the degree of name recognition in clear and distinct geographical areas. While our empirical strategy had drawbacks preventing a test of the relative sizes of campaign vs. incumbency effects, survey evidence uncovers patterns of name familiarity consistent with the personal vote. Voters in the district are more familiar with their representative than those outside. And familiarity is strongest in areas that remained inside the district after the map was redrawn, than in areas that migrated to the next district.  

The paper proceeds thus. Section 1 elaborates the electoral connection and notions of static and non-static ambition among politicians. Section 2 describes the Mexican reform, highlighting institutional limitations that might render reelection meaningless. Section 3 develops the procedure to separate campaign from incumbency effects by means of redistricting. Section 4 presents the survey and a multivariate model of name recognition. Section 5 concludes.  

----

Our investigation of name familiarity in Coahuila has achieved a number of things. We discussed how the electoral connection, by inducing politicians with static ambition to cultivate a personal vote by trading in particularistic goods, should heighten levels of name recognition among constituents. We also saw admonitions (skeptics' warning) of minimal effects in Mexico, as reformers chose to keep parties' firm grip on nominations in place and politicans in Latin America have proverbial lack of static ambition. So are there signs of an electoral connection in Coahuila's 2017 state legislative races, where incumbents were allowed to seek consecutive reelection for the first time in over eight decades? On top of admonitions (warnings), differentials in name familiarity could be due to campaigns, not the personal vote. 

The paper develops a method to overcome this obstacle by exploiting redistricting. Changes in the electoral geography are means to separate incumbency from campaign effects in name recognition. We set forth resolving these issues empirically with an original pre-election survey in the state. If unable to totally rule out the effect of campaigns, the public opinion evidence we produce does reveal statistically significant and substantive total effects in name recognition consistent with the personal vote. This is a noteworthy and important finding. 



  Lily Gutiérrez Burciaga     
  Georgina Cano Torralva     
  Sonia Villarreal Pérez     
  Javier Díaz González       
  Ana Isabel Durán Piña      
  Lencho Siller Linaje






\begin{table}
\centering
\begin{tabular}{cccc}
Variable                      & Statewide        & $\texttt{father}$ & $\texttt{son}$ \\ \hline
    \mc{3}{l}{~~A. \emph{Static ambition SMD$\rightarrow$SMD}} \\ \hdashline \\[-1.8ex] 
$\texttt{recognizeJavier}$    &  0.082    &  0.286  & 0.304  \\
$\texttt{recognizeLily}$      &  0.076    &  0.393  & 0.393  \\
$\texttt{recognizeGina}$      &  0.085    &  0.729  & 0.729  \\
    \mc{3}{l}{~~B. \emph{Progressive ambition SMD$\rightarrow$municipio}}  \\ \hdashline \\[-1.8ex]
$\texttt{recognizeLencho}$    &  0.066    &  0.300  & 0.500  \\
$\texttt{recognizeSonia}$     &  0.082    &  0.518  & 0.518  \\
$\texttt{recognizeAnaIsabel}$ &  0.068    &  0.589  & 0.738  \\ 
    \mc{3}{l}{~~C. \emph{Progressive ambition PR$\rightarrow$municipio}}   \\ \hdashline  \\[-1.8ex]
$\texttt{recognizeArmando}$   &  0.036    & 0.036   & 0.095  \\
$\texttt{recognizeLariza}$    &  0.063    & 0.063   & 0.048  \\
$\texttt{recognizeLeonel}$    &  0.056    & 0.056   & 0.405  \\ \hline
\end{tabular}
\caption{Mean name recognition ($N$ in parentheses)}\label{T:dvmean}
\end{table}




\begin{table}
  \centering
  \begin{tabular}{llrrrrrrr}
                & District/   &  \mc{4}{c}{Respondents}   & \mc{3}{c}{Mean DV}  \\ 
    Incumbent   & municipio   &  $l$ & $r$ & $g$& $n$   & state & father & son  \\ \hline
    \hline \\[-1.8ex] 
    \mc{7}{l}{~~A. \emph{Static ambition (SMD$\rightarrow$SMD)}} \\ \hdashline
  Javier PRI    & Saltillo    &  14 &  56 &  0  & 938 &  0.082    &  0.286  & 0.304 \\
  Lily PRI      & R. Arispe   &   0 &  56 &  0  & 952 &  0.076    &  0.393  & 0.393 \\
  Gina PRI      & Acuña       &   0 &  70 &  0  & 938 &  0.085    &  0.729  & 0.729 \\
  \\[-1.8ex] 
    \mc{7}{l}{~~B. \emph{Progressive ambition (SMD$\rightarrow$municipio)}} \\ \hdashline
  Lencho PRI    & Frontera    &  42 &  28 &  0  & 938 &  0.066    &  0.300  & 0.500 \\
  Sonia PRI     & P. Negras   &   0 &  56 &  0  & 952 &  0.082    &  0.518  & 0.518 \\
  AnaIsabel PRI & San Pedro   &  14 &  42 &  0  & 952 &  0.068    &  0.589  & 0.738 \\ 
  \\[-1.8ex]
    \mc{7}{l}{~~C. \emph{Progressive ambition (PR$\rightarrow$municipio)}} \\ \hdashline
  Armando PAN   & Frontera    & 966 &  42 &  0  &   0 &  0.036    & 0.036   & 0.095 \\
  Lariza PAN    & P. Negras   & 966 &  42 &  0  &   0 &  0.063    & 0.063   & 0.048 \\
  Leonel PPC    & Matamoros   & 966 &  42 &  0  &   0 &  0.056    & 0.056   & 0.405 \\
  \\[-1.8ex] \hline \hline
  \end{tabular}
  \caption{Incumbents and their terrain. Members with static ambition---from a single member district (SMD) running for a SMD---are distinguished from those with two types of progressive ambition---to a municipality from a SMD and from a PR seat. The first set of $l$, $r$, $g$, $n$ reports the number of electoral secciones (of 1,710 total in the state) in each category of terrain. The second reports the number of respondents sampled (out of 1,008) in each terrain category.}\label{T:terrenos}
\end{table}

