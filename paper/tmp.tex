
If the electoral connection presses politicians with static ambition to cultivate a personal vote by delivering particularistic goods, this should be detectable in heightened levels of name recognition among constituents.

We have presented the case of Coahuila state legislative races in 2017, allowing incumbents to seek consecutive reelection for the first time in over eight decades.

Reformers gave parties a prominent role in nominations, making skeptics doubt that the electoral connection might make any differnece in Mexico. We set forth attempting to find evidence of a personal vote with a pre-election survey in the state. 

On top of this lies the possibility that differentials in name familiarity could be due to campaigns, not the personal vote.

Redistricting 

Few ambitious

If unable to totally rule out the effect of campaigns, our survey does reveal statistically significant and substantive total effects in name recognition consistent with the personal vote. This is an important finding. 



  Lily Gutiérrez Burciaga     
  Georgina Cano Torralva     
  Sonia Villarreal Pérez     
  Javier Díaz González       
  Ana Isabel Durán Piña      
  Lencho Siller Linaje






\begin{table}
\centering
\begin{tabular}{cccc}
Variable                      & Statewide        & $\texttt{father}$ & $\texttt{son}$ \\ \hline
    \mc{3}{l}{~~A. \emph{Static ambition SMD$\rightarrow$SMD}} \\ \hdashline \\[-1.8ex] 
$\texttt{recognizeJavier}$    &  0.082    &  0.286  & 0.304  \\
$\texttt{recognizeLily}$      &  0.076    &  0.393  & 0.393  \\
$\texttt{recognizeGina}$      &  0.085    &  0.729  & 0.729  \\
    \mc{3}{l}{~~B. \emph{Progressive ambition SMD$\rightarrow$municipio}}  \\ \hdashline \\[-1.8ex]
$\texttt{recognizeLencho}$    &  0.066    &  0.300  & 0.500  \\
$\texttt{recognizeSonia}$     &  0.082    &  0.518  & 0.518  \\
$\texttt{recognizeAnaIsabel}$ &  0.068    &  0.589  & 0.738  \\ 
    \mc{3}{l}{~~C. \emph{Progressive ambition PR$\rightarrow$municipio}}   \\ \hdashline  \\[-1.8ex]
$\texttt{recognizeArmando}$   &  0.036    & 0.036   & 0.095  \\
$\texttt{recognizeLariza}$    &  0.063    & 0.063   & 0.048  \\
$\texttt{recognizeLeonel}$    &  0.056    & 0.056   & 0.405  \\ \hline
\end{tabular}
\caption{Mean name recognition ($N$ in parentheses)}\label{T:dvmean}
\end{table}




\begin{table}
  \centering
  \begin{tabular}{llrrrrrrr}
                & District/   &  \mc{4}{c}{Respondents}   & \mc{3}{c}{Mean DV}  \\ 
    Incumbent   & municipio   &  $l$ & $r$ & $g$& $n$   & state & father & son  \\ \hline
    \hline \\[-1.8ex] 
    \mc{7}{l}{~~A. \emph{Static ambition (SMD$\rightarrow$SMD)}} \\ \hdashline
  Javier PRI    & Saltillo    &  14 &  56 &  0  & 938 &  0.082    &  0.286  & 0.304 \\
  Lily PRI      & R. Arispe   &   0 &  56 &  0  & 952 &  0.076    &  0.393  & 0.393 \\
  Gina PRI      & Acuña       &   0 &  70 &  0  & 938 &  0.085    &  0.729  & 0.729 \\
  \\[-1.8ex] 
    \mc{7}{l}{~~B. \emph{Progressive ambition (SMD$\rightarrow$municipio)}} \\ \hdashline
  Lencho PRI    & Frontera    &  42 &  28 &  0  & 938 &  0.066    &  0.300  & 0.500 \\
  Sonia PRI     & P. Negras   &   0 &  56 &  0  & 952 &  0.082    &  0.518  & 0.518 \\
  AnaIsabel PRI & San Pedro   &  14 &  42 &  0  & 952 &  0.068    &  0.589  & 0.738 \\ 
  \\[-1.8ex]
    \mc{7}{l}{~~C. \emph{Progressive ambition (PR$\rightarrow$municipio)}} \\ \hdashline
  Armando PAN   & Frontera    & 966 &  42 &  0  &   0 &  0.036    & 0.036   & 0.095 \\
  Lariza PAN    & P. Negras   & 966 &  42 &  0  &   0 &  0.063    & 0.063   & 0.048 \\
  Leonel PPC    & Matamoros   & 966 &  42 &  0  &   0 &  0.056    & 0.056   & 0.405 \\
  \\[-1.8ex] \hline \hline
  \end{tabular}
  \caption{Incumbents and their terrain. Members with static ambition---from a single member district (SMD) running for a SMD---are distinguished from those with two types of progressive ambition---to a municipality from a SMD and from a PR seat. The first set of $l$, $r$, $g$, $n$ reports the number of electoral secciones (of 1,710 total in the state) in each category of terrain. The second reports the number of respondents sampled (out of 1,008) in each terrain category.}\label{T:terrenos}
\end{table}

