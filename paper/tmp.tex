
Elementos para reescribir esto 8abril2022 (post mpsa):
- en presentación MPSA flye bien un framing ms sólido. Centra el interés más balanceado en 2017/coahuila y separación. Usar esto en la intro. Para coahuila, la pregunta es si el efecto es atribuible o no a la reelección, con campaña la alternativa obvia. Hence the need for a method.
- Como sólo hay 3 incumbents de interés, debo introducir los smd->mun y los RP->mun como elementos secundarios. Reglas de codificación de l,r,g,n.
- Faltan los descriptivos de la encuesta. 




If we could find that voters in the district are generally familiar with their reelecting deputy to a degree not seen outside the district X does not fully answer the empirical question. Is X due to personal vote? Or is X due to campaign? 

If we could find a degree of name recognition among voters in the district unseen outside the district it still would not fully answer the empirical question. Is the finding due to the personal vote, as we argue in section 1? Or are voters familiar simply because of the campaign itself, which happens simultaneously and inevitably if the incumbent in on the ballot? Redistricting helps answer.

We examine name recognition in Coahuila. Discovery that voters in the district are generally familiar with their reelecting deputy to a degree not seen outside the district, this could be due to two forces. Is the discovery attributable to the personal vote, as we argue in section 1? Or are voters familiar due to the campaign itself, which inevitably happens simultaneously if the incumbent in on the ballot?  


Take advantage of redistricting to compare name familiarity among geographical groups of voters who either moved into, moved out of, or remained in the district. 

