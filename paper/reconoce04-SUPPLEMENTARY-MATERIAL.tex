\documentclass[letter,12pt]{article}
\usepackage[letterpaper,right=1.25in,left=1.25in,top=1in,bottom=1in]{geometry}
\usepackage{setspace}

\usepackage[utf8]{inputenc}   % allows input of special characters from keyboard (input encoding)
\usepackage[T1]{fontenc}      % what fonts to use when printing characters       (output encoding)
\usepackage{amsmath}          % facilitates writing math formulas and improves the typographical quality of their output
\usepackage[hyphens]{url}     % adds line breaks to long urls
\renewcommand{\UrlFont}{\ttfamily\small} % shrinks url font 1 step down
\usepackage[pdftex]{graphicx} % enhanced support for graphics
\usepackage{tikz}             % Easier syntax to draw pgf files (invokes pgf automatically)
\usetikzlibrary{arrows}

\usepackage{mathptmx}           % set font type to Times
\usepackage[scaled=.90]{helvet} % set font type to Times (Helvetica for some special characters)
\usepackage{courier}            % set font type to Times (Courier for other special characters)

\usepackage{rotating}         % sideway tables and figures that take a full page
\usepackage{caption}          % allows multipage figures and tables with same caption (\ContinuedFloat)

\usepackage{dcolumn}          % needed for apsrtable and stargazer tables from R to compile
\usepackage{arydshln}         % dashed lines in tables (hdashline, cdashline{3-4}, 
                              %see http://tex.stackexchange.com/questions/20140/can-a-table-include-a-horizontal-dashed-line)
                              % must be loaded AFTER dcolumn, 
                              %see http://tex.stackexchange.com/questions/12672/which-tabular-packages-do-which-tasks-and-which-packages-conflict

\usepackage{amssymb}          % has nicer empty set \varnothing, among much much more

%FOR SPANISH FORMATTING (HYPHENATION ETC.)
%% \usepackage[spanish]{babel}
%% \addto\captionsspanish{\renewcommand{\figurename}{Diagrama}} % cambia Figura por Diagrama

\usepackage[longnamesfirst, sort]{natbib}\bibpunct[]{(}{)}{,}{a}{}{;} % handles biblio and references 
%% \AtBeginDocument{\renewcommand\harvardand{y}} % change 'author and author' by Spanish 'author y author'

\newcommand{\mc}{\multicolumn}

%% TO ADD NOTES IN TEXT, PUT % BEFORE THE ONE YOU WANT DISBALED
%\usepackage[disable]{todonotes}                            % notes not showed
\usepackage[colorinlistoftodos, textsize=small]{todonotes} % show notes
\newcommand{\eric}[1]{\todo[color=red!15, inline]{\textbf{Eric:} #1}}
\newcommand{\alex}[1]{\todo[color=green!15, inline]{\textbf{Alejandro:} #1}}

% Format epigraph
\usepackage{epigraph}
\setlength\epigraphwidth{.8\textwidth}
\setlength\epigraphrule{0pt} % no rule

% multicolumns in appendix
\usepackage{multicol}

% change text color
\usepackage{xcolor}

\begin{document}

% \title{The removal of single-term limits, redistricting, and name recognition in Coahuila's state races}
% \title{Redistricting and the separation of incumbency and campaign effects: name recognition in Coahuila}
\title{The Personal Vote in Mexico: \\ Supplementary material}
\author{Eric Magar  \\ \url{emagar@itam.mx} \and
        Alejandro Moreno \\ \url{amoreno@itam.mx} 
}
\date{\today}
\maketitle

% \begin{center} \textbf{$\rightarrow$~~Preliminary draft~~$\leftarrow$} \\ (please inquire for new version)  \end{center}

%\onehalfspacing
%\doublespacing

\section{Online appendix}\label{appendix}

\subsection{Descriptive statistics: survey}

Unlike absolute frequencies In the summaries below, relative frequencies are parenthesized (reporting percentages). All missing answers are reported (NS/NC = ``No Answer/Don't Know''), so totals add to 1,008 respondents for all survey questions. Percentages may not add to 100 exactly due to rounding. The summary for `Localidad' illustrates, afterwards labels are omitted for economy. Section \ref{s:tran} includes an English translation of the survey questions about consecutive reelection. 

\begin{scriptsize}
\begin{verbatim}
Localidad
   Urbana         Rural         Mixta        Total
  N    (%)      N    (%)      N    (%)      N    (%)
----------------------------------------------------
840   (83)     70    (7)     98   (10)   1008  (100)

Municipio (% only)
            ABASOLO               ACUÑA             ALLENDE             ARTEAGA 
             (0.00)              (5.56)              (0.00)              (0.00) 
            CANDELA            CASTAÑOS      CUATROCIENEGAS            ESCOBEDO 
             (0.00)              (1.39)              (1.39)              (0.00) 
FRANCISCO I. MADERO            FRONTERA      GENERAL CEPEDA            GUERRERO 
             (1.39)              (2.78)              (1.39)              (0.00) 
            HIDALGO             JIMENEZ              JUAREZ            LAMADRID 
             (0.00)              (1.39)              (0.00)              (0.00) 
          MATAMOROS            MONCLOVA             MORELOS             MUZQUIZ 
             (4.17)              (6.94)              (0.00)              (2.78) 
          NADADORES                NAVA              OCAMPO              PARRAS 
             (0.00)              (1.39)              (0.00)              (1.39) 
     PIEDRAS NEGRAS            PROGRESO        RAMOS ARIZPE             SABINAS 
             (5.56)              (0.00)              (2.78)              (2.78) 
         SACRAMENTO            SALTILLO    SAN BUENAVENTURA SAN JUAN DE SABINAS 
             (0.00)             (26.39)              (1.39)              (1.39) 
          SAN PEDRO       SIERRA MOJADA             TORREON              VIESCA 
             (4.17)              (0.00)             (23.61)              (0.00) 
        VILLA UNION            ZARAGOZA               Total
             (0.00)              (0.00)            (100.00)

Sección electoral (N only)
   4    8   10   84  119  141  192  197  214  242  258  263  269  325  328  370 
-------------------------------------------------------------------------------
  14   14   14   14   14   14   14   14   14   14   14   14   14   14   14   14 
===============================================================================
 378  411  454  473  508  550  580  586  617  627  653  661  692  712  734  737 
-------------------------------------------------------------------------------
  14   14   14   14   14   14   14   14   14   14   14   14   14   14   14   14 
===============================================================================
 788  800  816  843  847  869  871  897  905  907  920  922  932  977  979 1042 
-------------------------------------------------------------------------------
  14   14   14   14   14   14   14   14   14   14   14   14   14   14   14   14 
===============================================================================
1068 1109 1145 1157 1173 1184 1214 1221 1263 1272 1303 1314 1316 1351 1377 1388 
-------------------------------------------------------------------------------
  14   14   14   14   14   14   14   14   14   14   14   14   14   14   14   14 
===============================================================================
1402 1447 1449 1464 1469 1623 1678 1703  Total
-------------------------------------------------------------------------------
  14   14   14   14   14   14   14   14   1008

Congressional district
       1        2        3        4        5        6        7      Total
-------------------------------------------------------------------------
140 (14) 154 (15) 154 (15) 154 (15) 140 (14) 154 (15) 112 (11) 1008 (100) 

State assembly district
     1      2      3      4      5      6      7      8      9     10     11     12     13  
------------------------------------------------------------------------------------------
70 (7) 70 (7) 70 (7) 70 (7) 56 (6) 56 (6) 56 (6) 70 (7) 84 (8) 42 (4) 42 (4) 56 (6) 84 (8) 
==========================================================================================
     14     15     16      Total
--------------------------------
 56 (6) 84 (8) 42 (4) 1008 (100)

(Si tiene credencial para votar vigente) ¿Está registrada en este domicilio o en otro?
   En este        En otro
-------------------------
813   (81)     195   (19) 

Sexo
   Hombre        Mujer
----------------------
502  (50)    506  (50)
 
Edad (grouped)
 [18,28)   [28,38)   [38,48)   [48,58)   [58,68)    [68,+)
----------------------------------------------------------         
211 (21)  205 (20)  203 (20)  180 (18)  117 (12)   92  (9)

 
1 En su opinión, ¿cuál es el principal problema que hay actualmente en el Estado de 
Coahuila? (ANOTAR TEXTUAL)                   (%)
------------------------------------------------
                                  Campo    (0.2) 
                             Corrupción   (13.7) 
                              Desempleo   (12.4) 
                               Economía    (6.5) 
                              Educación    (0.8) 
            Falta de buenos gobernantes    (0.9) 
                    Fenómenos naturales    (0.1) 
Inflación/alza de precios/precios altos    (0.8) 
                    Inseguridad pública   (45.0) 
                           Narcotráfico    (1.0) 
                                Ninguno    (1.0) 
                         Obras públicas    (0.4) 
                                Pobreza    (3.1) 
                    Problemas políticos    (0.5) 
                        Gobierno de epn    (0.3) 
                     Problemas sociales    (1.1) 
                                  Salud    (0.9) 
                     Servicios públicos    (9.7) 
                                  Todos    (1.0) 
                                  Otros    (0.4) 
              Falta de ayuda a la gente    (0.2) 
          Demasiados programas sociales    (0.1) 

2 Por lo general, ¿cuánto le interesa la política? (LEER)
     Mucho         Algo         Poco         Nada        NS/NC
--------------------------------------------------------------
122 (12.1)   259 (25.7)   324 (32.1)   301 (29.9)     2  (0.2) 

3 ¿Sabe cuándo son las próximas elecciones para Gobernador del estado? 
(NO LEER: 4 DE JUNIO 2017)
Sabe completa    Incompleta        NS/NC 
----------------------------------------
    719  (71)    160   (16)    129  (13) 

4 Del 0 a 10, donde 0 es nada probable y 10 muy probable, ¿qué tan probable es que 
usted vote en las próximas elecciones para gobernador?
        0         1         2         3         4         5 
---------------------------------------------------------------
 64 (6.3)  37 (3.7)  21 (2.1)  25 (2.5)  14 (1.4)  90 (8.9)
===============================================================
        6         7          8         9          10      NS/NC
----------------------------------------------------------------
 36 (3.6)  50 (5.0) 113 (11.2)  49 (4.9)  505 (50.1)     4 (0.4) 
         
5 (USAR BOLETA 1) Si hoy hubiera elecciones para Gobernador del estado, ¿por 
quién votaría                            N       (%)
----------------------------------------------------
               Guillermo Anaya, PAN    194    (19.2) 
             Miguel A. Riquelme,PRI    238    (23.6) 
              Mary T. Guajardo, PRD     15     (1.5) 
                  José A. Pérez, PT     15     (1.5) 
           Miguel A. Riquelme, PVEM     16     (1.6) 
               Guillermo Anaya, UDC     14     (1.4) 
          Miguel A. Riquelme, PANAL      8     (0.8) 
            Miguel A. Riquelme, PSI      4     (0.4) 
               Guillermo Anaya, PPC      2     (0.2) 
  Miguel A. Riquelme, Partido Jóven      9     (0.9) 
            Miguel A. Riquelme, PRC      3     (0.3) 
            Miguel A. Riquelme, PCP      4     (0.4) 
           Armando Guadiana, Morena    102    (10.1) 
  Guillermo Anaya, Encuentro Social      7     (0.7) 
     Javier Guerrero, Independiente     45     (4.5) 
Luis Horacio Salinas, Independiente     12     (1.2) 
                      No registrado      5     (0.5) 
                               Nulo     88     (8.7) 
                            Ninguno     52     (5.2) 
                              NS/NC    175    (17.4) 
                              TOTAL   1008   (100.0)

6 ¿Usted ya decidió definitivamente por quién votar para gobernador, 
tiene idea pero podría cambiar o aún no decide su voto?
                                N     (%)
-----------------------------------------
Ya decidió definitivamente    558    (55)
Tiene idea, podría cambiar    150    (15)
             Aún no decide    246    (24)
                     NS/NC     54     (5) 

7 ¿Cuál es su opinión acerca de los siguientes personajes políticos: muy buena, buena, mala, 
muy mala,... o no lo conoce suficiente para opinar? (LEER Y ROTAR NOMBRES)

a Guillermo Anaya Llamas       
    Muy                             Muy    Ni buena               No lo  
  buena     Buena       Mala       mala     ni mala      NS/NC   conoce       Total
-----------------------------------------------------------------------------------
 41 (4)  260 (26)   191 (19)   106 (11)    260 (26)   100 (10)   50 (5)  1008 (100)  
 
b Miguel Ángel Riquelme        
    Muy                             Muy    Ni buena               No lo  
  buena     Buena       Mala       mala     ni mala     NS/NC    conoce      Total
----------------------------------------------------------------------------------
 45 (4)  224 (22)   225 (22)   187 (19)    195 (19)    93 (9)    39 (4)  1008 (100) 

c Mary Telma Guajardo Villareal
    Muy                             Muy    Ni buena               No lo  
  buena     Buena       Mala       mala     ni mala     NS/NC    conoce      Total
----------------------------------------------------------------------------------
  3 (0)    77 (8)    96 (10)     64 (6)    162 (16)   254 (25) 352 (35)  1008 (100)

d José Ángel Pérez Hernández   
    Muy                             Muy    Ni buena               No lo  
  buena     Buena       Mala       mala     ni mala     NS/NC    conoce      Total
----------------------------------------------------------------------------------
 15 (1)    95 (9)     95 (9)     76 (8)    164 (16)   274 (27) 289 (29)  1008 (100)

e Armando Guadiana Tijerina    
    Muy                             Muy    Ni buena               No lo  
  buena     Buena       Mala       mala     ni mala     NS/NC    conoce      Total
----------------------------------------------------------------------------------
 21 (2)  159 (16)   88   (9)   71   (7)   180  (18)  210 (21)  279 (28)  1008 (100)

f Javier Guerrero García       
    Muy                             Muy    Ni buena               No lo  
  buena     Buena       Mala       mala     ni mala     NS/NC    conoce      Total
----------------------------------------------------------------------------------
 16 (2)  137 (14)     74 (7)     61 (6)    180 (18)  256 (25)  284 (28)  1008 (100)

g Luis Horacio Salinas Valdez  
    Muy                             Muy    Ni buena               No lo  
  buena     Buena       Mala       mala     ni mala     NS/NC    conoce      Total
----------------------------------------------------------------------------------
  6 (1)    56 (6)     71 (7)     60 (6)    138 (14)   284 (28) 393 (39)  1008 (100) 

8 ¿Si la elección para gobernador solamente fuera entre Guillermo Anaya y Miguel 
Riquelme, ¿por quién votaría usted?                          N   (%)
--------------------------------------------------------------------
                    Guillermo Anaya del PAN-UDC-PPC-PES   334   (33) 
Miguel Ángel Riquelme del PRI-PVEM-PANAL-PSI-PJ-PRC-PCP   314   (31) 
                                                Ninguno   273   (27) 
                                                  NC/NC    87    (9) 

9 ¿Quién cree que gane la elección para gobernador? (LEER)
Guillermo Anaya del PAN  Miguel Ángel Riquelme del PRI    Otro       NS/NC
--------------------------------------------------------------------------
               258 (26)                       487 (48)    0 (0)   263 (26) 

10 De los siguientes asuntos que le voy a leer, dígame por favor cuál es el 
más importante que debe atender el próximo gobernador del estado: 
(LEER)                   N   (%)
--------------------------------
        Inseguridad    319  (32)
            Pobreza    193  (19)
            Empleos    170  (17)
         Corrupción    170  (17)
          Educación     41   (4)
     Medio ambiente     10   (1)
La deuda del estado     66   (7)
               Otro      0   (0)
              NS/NC     39   (4)
 
11 (USAR BOLETA 2) Si hoy hubiera elecciones para Diputados Locales, ¿por cuál partido 
votaría usted?
     PAN      PRI     PRD      PT    PVEM     UDC     MC   PANAL    PSI    PPC 
------------------------------------------------------------------------------
206 (20) 264 (26)  16 (2)  20 (2)  17 (2)  23 (2)  6 (1)  13 (1)  1 (0)  7 (1) 
==============================================================================
     PJ     PRC     PCP  MORENA    PES  Independiente       <NA>       Total
----------------------------------------------------------------------------
 15 (1)   3 (0)   6 (1)  88 (9)  5 (0)         35 (3)   283 (28)  1008 (100)

12 (USAR BOLETA 3) Si hoy hubiera elecciones para Presidente Municipal, ¿por cuál partido 
votaría usted?
     PAN      PRI     PRD      PT    PVEM     UDC     MC   PANAL    PSI    PPC 
------------------------------------------------------------------------------
218 (22) 287 (28)  17 (2)  14 (1)  11 (1)  22 (2)  8 (1)   5 (0)  1 (0)  9 (1) 
==============================================================================
     PJ     PRC     PCP  MORENA    PES  Independiente       <NA>       Total
----------------------------------------------------------------------------
 13 (1)   2 (0)   6 (1)  78 (8)  5 (0)          38 (4)  274 (27)  1008 (100)

13 ¿Votó usted en las elecciones para gobernador en julio de 2011? (SÍ) 
¿Por quién votó usted? (LEER OPCIONES)     N     (%)
----------------------------------------------------
      Guillermo Anaya Llamas, PAN-UDC    211    (21) 
Rubén Moreira, PRI-PVEM-PANAL-PPC-PSI    367    (36) 
                                 Otro      0     (0) 
                        No registrado     22     (2) 
                                 Nulo     21     (2) 
                              No votó    255    (25) 
                                NS/NC    132    (13) 

14 En general, ¿usted aprueba o desaprueba el trabajo que Rubén Moreira está haciendo 
como Gobernador del estado? (INSISTIR): ¿APRUEBA/DESAPRUEBA mucho o algo?
Aprueba mucho    Aprueba algo   Desaprueba algo   Desaprueba mucho    NS/NC 
---------------------------------------------------------------------------
       88 (9)        294 (29)          233 (23)           357 (35)   36 (4) 

15 En general, ¿está satisfecho o insatisfecho con la manera en que marchan las cosas 
en el estado? (INSISTIR: ¿Muy o algo?) (5=NS/NC)
Muy satisfecho    Algo satisfecho    Algo satisfecho    Muy insatisfecho     NS/NC
----------------------------------------------------------------------------------
        52 (5)           329 (33)           318 (32)            297 (29)    12 (1) 

16 En general, ¿usted aprueba o desaprueba el trabajo que Enrique Peña Nieto está 
haciendo como Presidente de la República? (INSISTIR): ¿APRUEBA/DESAPRUEBA mucho o algo?
Aprueba mucho    Aprueba algo   Desaprueba algo   Desaprueba mucho    NS/NC 
---------------------------------------------------------------------------
       72 (7)        199 (20)          184 (18)           531 (53)   22 (2) 

17 ¿Cómo calificaría en estos momentos... (LEER):? muy bien, bien, mal o muy mal?

a La situación económica del estado
Muy bien      Bien       Mal    Muy mal   Ni bien ni mal    NS/NC 
-----------------------------------------------------------------
  10 (1)  164 (16)  319 (32)   330 (33)         182 (18)    3 (0) 

b Su situación económica familiar
Muy bien      Bien       Mal    Muy mal   Ni bien ni mal    NS/NC 
-----------------------------------------------------------------
  20 (2)  344 (34)  231 (23)   130 (13)         282 (28)    1 (0) 

c La seguridad pública en la comunidad donde vive
Muy bien      Bien       Mal    Muy mal   Ni bien ni mal    NS/NC 
-----------------------------------------------------------------
  33 (3)  290 (29)  277 (27)   223 (22)         177 (18)    8 (1) 

18 Generalmente, ¿usted se considera priista, panista, perredista morenista? (INSISTIR): 
¿Se considera muy o algo?
       Priista        Panista       Perredista     Morenista
     muy     algo    muy   algo    muy    algo     muy    algo   NS/NC    Otro  Ninguno
---------------------------------------------------------------------------------------
151 (15) 101 (10) 70 (7) 43 (4)  9 (1)   3 (0)  23 (2)  22 (2)   0 (0) 538 (53)  48 (5) 

19 (TARJETA 1) En política la gente habla de “la izquierda” y “la derecha”. En general, 
¿cómo colocaría usted sus puntos de vista en esta escala, donde 1 es izquierda y 10 es 
derecha? También puede escoger un punto intermedio.
       1      2      3      4        5      6      7      8      9       10    NS/NC 
------------------------------------------------------------------------------------
115 (11) 29 (3) 52 (5) 48 (5) 214 (21) 87 (9) 49 (5) 79 (8) 32 (3) 127 (13) 176 (17) 

20 ¿Está usted a favor, en contra o le es indiferente la reelección consecutiva de 
legisladores?
 A favor  En contra  Le es indiferente   NS/NC 
----------------------------------------------
121 (12)   511 (51)           320 (32)  56 (6) 

21 El 3 de abril iniciaron las campañas para renovar el Congreso del Estado. Si yo le 
preguntara los nombres de los candidatos a diputado en este distrito, ¿usted me podría 
decir todos los nombres, algunos nombres o no recuerda ningún nombre en este momento?
Todos   Algunos  No recuerda  No contestó 
-----------------------------------------
9 (1)  144 (14)     783 (78)       72 (7) 

22 Ahora piense por favor en los diputados locales actuales. Si yo le preguntara las 
cosas que ha hecho su diputado por esta comunidad, ¿usted podría mencionarme muchas cosas, 
algunas, diría que no hizo nada o no recuerda en este momento?
Muchas   Algunas   No hizo nada   No recuerda    NS/NC
------------------------------------------------------
18 (2)  217 (22)       495 (49)      266 (26)   12 (1) 

23 Si su actual diputado compitiera para buscar la reelección, ¿usted votaría por él 
o no votaría por él?
Sí votaría por él   No votaría por él      NS/NC 
------------------------------------------------
         156 (15)            731 (73)   121 (12) 

24 Con base en el trabajo realizado por su actual diputado, ¿cree que merecería ser 
reelecto en su cargo o no?
      Sí        No        NC 
----------------------------
158 (16)  751 (75)   99 (10) 

25 Le voy a leer unos nombres, para cada uno, ¿podría decirme si le es muy conocido, 
algo conocido, poco o nada conocido?

a Javier Díaz González     
Muy conocido     Algo     Poco   Nada conocido    NS/NC 
-------------------------------------------------------
      17 (2)   30 (3)   36 (4)        889 (88)   36 (4) 

b Lily Gutiérrez Burciaga  
Muy conocido     Algo     Poco   Nada conocido    NS/NC 
-------------------------------------------------------
      14 (1)   34 (3)   29 (3)        895 (89)   36 (4) 

c Georgina Cano Torralva   
Muy conocido     Algo     Poco   Nada conocido    NS/NC 
-------------------------------------------------------
      22 (2)   40 (4)   24 (2)        884 (88)   38 (4) 

d Ana Isabel Durán         
Muy conocido     Algo     Poco   Nada conocido    NS/NC 
-------------------------------------------------------
      10 (1)   34 (3)   25 (2)        901 (89)   38 (4) 

e Sonia Villareal          
Muy conocido     Algo     Poco   Nada conocido    NS/NC 
-------------------------------------------------------
      20 (2)   41 (4)   22 (2)        888 (88)   37 (4) 

f Lariza Montiel           
Muy conocido     Algo     Poco   Nada conocido    NS/NC 
-------------------------------------------------------
      18 (2)   33 (3)   13 (1)        906 (90)   38 (4) 

g Armando Pruneda          
Muy conocido     Algo     Poco   Nada conocido    NS/NC 
-------------------------------------------------------
       6 (1)   20 (2)   10 (1)        933 (93)   39 (4) 

h Leonel Contreras Pámanes 
Muy conocido     Algo     Poco   Nada conocido    NS/NC 
-------------------------------------------------------
       6 (1)   25 (2)   25 (2)        912 (90)   40 (4) 

i Florencio "Lencho" Siller
Muy conocido     Algo     Poco   Nada conocido    NS/NC 
-------------------------------------------------------
       7 (1)   29 (3)   31 (3)        902 (89)   39 (4) 

26 En los últimos 12 meses, ¿usted o alguien de su familia... (LEER)

a Perdió su empleo o fuente de ingresos económicos?
Sí, usted   Sí, un familiar   Sí, ambos        No     NS/NC 
-----------------------------------------------------------
 141 (14)          212 (21)      17 (2)  635 (63)     3 (0) 

b Fue víctima de algún delito o un asalto?
Sí, usted   Sí, un familiar   Sí, ambos        No     NS/NC 
-----------------------------------------------------------
  97 (10)            93 (9)      20 (2)  796 (79)     2 (0) 

c Tuvo que dar alguna mordida
Sí, usted   Sí, un familiar   Sí, ambos        No     NS/NC 
-----------------------------------------------------------
  99 (10)            54 (5)      12 (1)  839 (83)     4 (0) 

27 Por lo general, ¿cuánto se entera de las noticias por medio de... (LEER), 
mucho, algo, poco o nada?

a Televisión
   Mucho       Algo       Poco       Nada      NS/NC 
----------------------------------------------------
413 (41)   242 (24)   228 (23)   123 (12)      2 (0) 

b Radio
   Mucho       Algo       Poco       Nada      NS/NC 
----------------------------------------------------
188 (19)   222 (22)   190 (19)   404 (40)      4 (0) 

c Periódico
   Mucho       Algo       Poco       Nada      NS/NC 
----------------------------------------------------
132 (13)   167 (17)   173 (17)   530 (53)      6 (1) 

d Pláticas con gente
   Mucho       Algo       Poco       Nada      NS/NC 
----------------------------------------------------
190 (19)   271 (27)   183 (18)   355 (35)      9 (1) 

e Internet
   Mucho       Algo       Poco       Nada      NS/NC 
----------------------------------------------------
274 (27)   149 (15)    99 (10)   474 (47)     12 (1) 

f Redes sociales
   Mucho       Algo       Poco       Nada      NS/NC 
----------------------------------------------------
278 (28)   153 (15)    82  (8)   482 (48)     13 (1) 

28 ¿Utiliza Facebook?
      Sí       No      NC
-------------------------
559 (55) 444 (44)   5 (0) 

29 ¿Utiliza Twitter?
      Sí       No      NC
-------------------------
138 (14) 868 (86)   2 (0) 

30 ¿Tiene Smartphone o teléfono inteligente?
      Sí       No      NC
-------------------------
562 (56) 441 (44)   5 (0) 

31 ¿Usted o alguien en su hogar es beneficiario de... (LEER)?

a Oportunidades/Prospera
Sí, usted    Sí, un familiar     Sí, ambos           No       NS/NC
-------------------------------------------------------------------
   94 (9)            98 (10)        10 (1)     797 (79)       9 (1) 

b Seguro Popular
Sí, usted    Sí, un familiar     Sí, ambos           No       NS/NC
-------------------------------------------------------------------
 162 (16)             84 (8)        67 (7)     688 (68)       7 (1) 

c Algún programa social del gobierno del estado
Sí, usted    Sí, un familiar     Sí, ambos           No       NS/NC
-------------------------------------------------------------------
 120 (12)             65 (6)        11 (1)     802 (80)      10 (1) 

32 Durante estas campañas electorales, ¿a usted o alguien en su hogar... (LEER)?

a Le han dado algún obsequio los partidos o candidatos
Sí, usted    Sí, un familiar     Sí, ambos           No       NS/NC
-------------------------------------------------------------------
 110 (11)             67 (7)        25 (2)     803 (80)       3 (0) 

b Ha asistido a eventos de los partidos o candidatos
Sí, usted    Sí, un familiar     Sí, ambos           No       NS/NC
-------------------------------------------------------------------
 135 (13)             52 (5)       22  (2)     796 (79)       3 (0) 

33 Si los candidatos a la Presidencia de la República en 2018 fueran los siguientes, 
¿por quién votaría usted? (LEER Y ROTAR)

[[NOT INCLUDED IN DATABASE???]]

34 ¿Cuál es su opinión acerca de los siguientes personajes políticos: muy buena, 
buena, mala, muy mala,... o no lo conoce suficiente para opinar? (LEER Y ROTAR NOMBRES)
a Andrés Manuel López Obrador
b Margarita Zavala
c Miguel Ángel Osorio Chong
d Humberto Moreira

[[NOT INCLUDED IN DATABASE???]]

35 Juntando el dinero que usted y otros miembros de su familia ganan al mes, 
¿diría que...? (LEER)
                            Les alcanza bien    207    (21) 
        Les alcanza con algunas dificultades    459    (46) 
                              No les alcanza    238    (24) 
No les alcanza y tienen grandes dificultades     99    (10) 
                                       NS/NC      5     (0) 

A ¿Hasta qué año o grado aprobó (pasó) en la escuela? ¿Cuál es su último grado de 
estudios? [NS/NC=9]
                           Ninguno    30    (3) 
                    Hasta primaria   234   (23) 
                        Secundaria   338   (34) 
       Preparatoria o bachillerato   226   (22) 
Normal/Carrera técnica o comercial    59    (6) 
          Universidad sin terminar    36    (4) 
             Universidad terminada    69    (7) 
       Posgrado/Maestría/Doctorado    15    (1) 
                             NS/NC     1    (0) 

B ¿Cuál es su principal ocupación, a qué se dedica usted? (ANOTAR TEXTUAL)
          Patrón/ Gerente/ Directivo Funcionario/ Empresario     7   (1) 
                                               Profesionista    48   (5) 
          Trabajos de oficina con cargo de jefe o supervisor     7   (1) 
        Trabajador de oficina bajo supervisión (oficinistas)     9   (1) 
                             Trabajador manual especializado    67   (7) 
                         Trabajador manual semi-espicalizado   191  (19) 
                          Trabajador manual no especializado    11   (1) 
                                         Trabajador agrícola    27   (3) 
       Comerciante: Ventas (cuando no menciona lo que vende)    56   (6) 
                                          Vendedor ambulante     1   (0) 
                                                    Empleado    99  (10) 
                                                 Desempleado    37   (4) 
                                        Jubilado/ Pensionado    63   (6) 
                                        Estudiante/ Becarios    36   (4) 
                                                 Ama de casa   343  (34) 
                                          No tiene actividad     3   (0) 
                                                 No contestó     3   (0) 

C ¿De qué religión es usted? (LEER Y ROTAR)
Católica   Cristiana/Evangélica/Protestante   Otra    NS/NC     Ateo
---------------------------------------------------------------------
 735 (73)             119 (12)                8 (1)  99 (10)   47 (5) 

D ¿Con qué frecuencia asiste usted a servicios religiosos? (LEER)
    Más de una       Una vez   Una vez    Sólo ocasiones       Nunca,
vez por semana    por semana    al mes        especiales   casi nunca   NS/NC
-----------------------------------------------------------------------------
    93  (9)        282  (28)  159 (16)         260  (26)     140  (14) 74 (7) 
\end{verbatim}
\end{scriptsize}

\subsection{Definitions and descriptive statistics: variables in the models}

\begin{itemize}
\item The $\texttt{recognize}$ variables (one for each of the six candidates analyzed) were coded with question 25 items. So $\texttt{recognizeJavier}_i$ equals 1 if respondent $i$ expressed much, some, or mild knowledge when told the name Javier Díaz González in item 25a; equals 0 otherwise. We proceeded likewise with items 25b (Lily), 25c (Gina), 25d (Ana Isabel), 25e (Sonia), and 25i (Lencho). 
\item $\texttt{delivered}_i$, coded with question 22, equals 1 if respondent $i$ answered that his/her state deputy brough many or some things to the community; equals 0 otherwise. 
\item $\texttt{interested}_i$, coded with question 2, equals 1 if respondent $i$ expressed much or some interest in politics; equals 0 otherwise.
\item $\texttt{handout}_i$, coded with question 32a, equals 1 if respondent $i$ answered that a party or candidate handed her/him or someone in the family a present; equals 0 if the answwer was no.
\item $\texttt{panista}_i$, coded with question 18, equals 1 if respondent $i$ answered strong or weak panista; equals 0 otherwise. $\texttt{priista}_i$, $\texttt{perredista}_i$, and $\texttt{morenista}_i$ coded likewise with the corresponding items. The reference category for these mutually exclusive indicators are respondents who identify with another party, with no party, or gave no answer.
\item The geographic indicators were coded by mapping $\texttt{sección}$ to the father and son district maps.
\end{itemize}

% \begin{table}
\scalebox{.75}{
\begin{tabular}{c|rr|rr|rr|rr|rr|rr|r}
\multicolumn{1}{c}{Variable} & 0 & \multicolumn{1}{r}{1} & $N$ \\ \hline
$\texttt{delivered}$  & 773 & 235 & 1008 \\
$\texttt{interested}$ & 627 & 381 & 1008 \\
$\texttt{smartphone}$ & 446 & 562 & 1008 \\
$\texttt{handout}$    & 803 & 202 & 1005 \\
$\texttt{panista}$    & 895 & 113 & 1008 \\
$\texttt{priista}$    & 756 & 252 & 1008 \\
$\texttt{morenista}$  & 963 & 45  & 1008 \\ \\ [-1.8ex]
 &\multicolumn{2}{r|}{Javier}&\multicolumn{2}{r|}{Lily}&\multicolumn{2}{r|}{Gina}&\multicolumn{2}{r|}{Lencho}&\multicolumn{2}{r|}{Sonia}&\multicolumn{2}{r|}{AnaIsabel}& \\
&0&1&0&1&0&1&0&1&0&1&0&1&$N$\\
$\texttt{recognize}$ (DV) & 925 & 83 & 931 & 77 & 922 & 86 & 941 & 67 & 925 & 83 & 939 & 69 & 1008 \\
% $\texttt{recognizeLily}$ & 931 & 77 & 1008 \\
% $\texttt{recognizeGina}$ & 922 & 86 & 1008 \\
% $\texttt{recognizeLencho}$ & 941 & 67 & 1008 \\
% $\texttt{recognizeSonia}$ & 925 & 83 & 1008 \\
% $\texttt{recognizeAnaIsabel}$ & 939 & 69 & 1008 \\
$\texttt{lost}$    & 994 & 14 & 1008 & & 1008 & & 966 & 42 & 1008 & & 994 &  14 & 1008 \\
% $\texttt{lostLily}$      & 1008 && 1008  \\
% $\texttt{lostGina}$      & 1008 && 1008 \\
% $\texttt{lostLencho}$    & 966 & 42 & 1008 \\
% $\texttt{lostSonia}$     & 1008 && 1008 \\
% $\texttt{lostAnaIsabel}$ & 994 &  14 & 1008 \\
$\texttt{retained}$    & 952 & 56 & 952 & 56 & 938 & 70 & 980 & 28 & 952 & 56 & 966 & 42 & 1008 \\
% $\texttt{retainedLily}$      & 952 & 56 & 1008 \\
% $\texttt{retainedGina}$      & 938 & 70 & 1008 \\
% $\texttt{retainedLencho}$    & 980 & 28 & 1008 \\
% $\texttt{retainedSonia}$     & 952 & 56 & 1008 \\
% $\texttt{retainedAnaIsabel}$ & 966 & 42 & 1008 \\
$\texttt{gained}$ (dropped)   & 1008 & & 1008 & & 1008 & & 1008 & & 1008 & & 1008 && 1008 \\
% $\texttt{gainedLily}$      & 1008 && 1008 \\
% $\texttt{gainedGina}$      & 1008 && 1008 \\
% $\texttt{gainedLencho}$    & 1008 && 1008 \\
% $\texttt{gainedSonia}$     & 1008 && 1008 \\
% $\texttt{gainedAnaIsabel}$ & 1008 && 1008 \\
$\texttt{nomans}$ (dropped)   &  70 & 938 &  56 & 952 &  70 & 938 &  70 & 938 &  56 & 952 &  56 & 952 & 1008  \\ \hline
% $\texttt{nomanLily}$      &  56 & 952 & 1008 \\
% $\texttt{nomanGina}$      &  70 & 938 & 1008 \\
% $\texttt{nomanLencho}$    &  70 & 938 & 1008 \\
% $\texttt{nomanSonia}$     &  56 & 952 & 1008 \\
% $\texttt{nomanAnaIsabel}$ &  56 & 952 & 1008 \\
\end{tabular}
}
% \caption{Variables in the models, frequencies} 
% \label{T:desc} 
% \end{table}


\subsection{Survey questions}\label{s:tran}
Thirteen items in the survey questionnaire involved reelection and name recognition (from question 20 to question 25.i) . We used questions 25.a--25.i to code our dependent variables. Responses much/some/little (\emph{mucho/algo/poco}) coded as 1 in the incumbent's name recognition indicator; 0 otherwise.

\begin{multicols}{2}

\begin{scriptsize}
\begin{verbatim}
20 Are you in favor, against or indifferent 
towards the consecutive reelection of 
lawmakers?

1) In favor 
2) Against
3) Indifferent
4) Don't know / No answer

21 On April 3, campaigns to renew the State 
Congress began. If I asked you the names of 
the candidates for deputy in this district, 
could you tell me all the names, some names 
or do not remember any names at this moment?

1) All 
2) Some
3) Don't remember
4) No answer

22 Now please think about the current local 
deputies. If I asked you the things your 
deputy has done for this community, could 
you mention many things, some, would you 
say he did nothing or do not remember at 
this moment? [5=NR/NA]

1) Many
2) Some
3) Did nothing
4) Don't remember

23 If your current deputy were running for 
reelection, would you vote for him or not 
vote for him?

1) Yes, I would vote for him/her
2) Would not vote for him/her
3) Don't known / No answer (DO NOT READ)

24 Based on the work done by your current 
deputy, do you think he/she would deserve 
to be reelected in his position or not?

[1=Yes; 2=No; 3= No answer]

25 I'm going to read you some names, for 
each one, could you tell me if he/she is 
well known, somewhat known, little known 
or not known at all?

[1= Well known; 2=Somewhat known; 
3= Little known; 4=Not known at all; 
5= DK/NA].

a Javier Díaz González     
b Lily Gutiérrez Burciaga  
c Georgina Cano Torralva   
d Ana Isabel Durán         
e Sonia Villareal          
f Lariza Montiel           
g Armando Pruneda          
h Leonel Contreras Pámanes 
i Florencio ``Lencho'' Siller
\end{verbatim}
\end{scriptsize}

% \columnbreak

% \begin{scriptsize}
% \begin{verbatim}
% 20 ¿Está usted a favor, en contra o le 
% es indiferente la reelección consecutiva 
% de legisladores?

% 1) A favor 2) En contra
% 3) Le es indiferente
% 4) NS/NC

% 21 El 3 de abril iniciaron las campañas 
% para renovar el Congreso del Estado. Si yo 
% le preguntara los nombres de los candidatos 
% a diputado en este distrito, ¿usted me 
% podría decir todos los nombres, algunos 
% nombres o no recuerda ningún nombre en
% este momento?

% 1) Todos 2) Algunos
% 3) No recuerda
% 4) No contestó

% 22 Ahora piense por favor en los diputados
% locales actuales. Si yo le preguntara las
% cosas que ha hecho su diputado por esta
% comunidad, ¿usted podría mencionarme muchas
% cosas, algunas, diría que no hizo nada o no
% recuerda en este momento? [5=NS/NC]

% 1) Muchas
% 2) Algunas
% 3) No hizo nada
% 4) No recuerda

% 23 Si su actual diputado compitiera para
% buscar la reelección, ¿usted votaría por
% él o no votaría por él?

% 1) Sí votaría por él
% 2) No votaría por él
% 3) NS/NC (NO LEER)

% 24 Con base en el trabajo realizado por 
% suactual diputado, ¿cree que merecería 
% ser reelecto en su cargo o no?

% [1=Sí; 2=No; 3=NC]

% 25 Le voy a leer unos nombres, para cada 
% uno, ¿podría decirme si le es muy conocido, 
% algo conocido, poco o nada conocido?

% [1=Muy conocido; 2=Algo; 3=Poco; 
% 4=Nada conocido; 5=NS/NC]

% a Javier Díaz González     
% b Lily Gutiérrez Burciaga  
% c Georgina Cano Torralva   
% d Ana Isabel Durán         
% e Sonia Villareal          
% f Lariza Montiel           
% g Armando Pruneda          
% h Leonel Contreras Pámanes 
% i Florencio ``Lencho'' Siller
% \end{verbatim}
% \end{scriptsize}

\end{multicols}

\subsection{Regression results}

See Table \ref{T:regs}.

% Table created by stargazer v.5.2 by Marek Hlavac, Harvard University. E-mail: hlavac at fas.harvard.edu
% Date and time: Wed, Feb 14, 2018 - 08:56:20 PM
% Requires LaTeX packages: dcolumn 
\begin{sidewaystable}[!htbp] \centering 
%%\begin{tabular}{l|rrr|rrr|rrr} 
    \scalebox{.85}{
\begin{tabular}{@{\extracolsep{5pt}}lD{.}{.}{-3} D{.}{.}{-3} D{.}{.}{-3} D{.}{.}{-3} D{.}{.}{-3} D{.}{.}{-3} D{.}{.}{-3} D{.}{.}{-3} D{.}{.}{-3} } 
%\\[-1.8ex]\hline 
%\hline \\[-1.8ex] 
\\[-1.8ex] & \multicolumn{1}{c}{(1)} & \multicolumn{1}{c}{(2)} & \multicolumn{1}{c}{(3)} & \multicolumn{1}{c}{(4)} & \multicolumn{1}{c}{(5)} & \multicolumn{1}{c}{(6)} & \multicolumn{1}{c}{(7)} & \multicolumn{1}{c}{(8)} & \multicolumn{1}{c}{(9)}\\ 
  & \multicolumn{1}{c}{Javier} & \multicolumn{1}{c}{Lily} & \multicolumn{1}{c}{Gina} & \multicolumn{1}{c}{Lencho} & \multicolumn{1}{c}{Sonia} & \multicolumn{1}{c}{A.Isabel} & \multicolumn{1}{c}{Armando} & \multicolumn{1}{c}{Lariza} & \multicolumn{1}{c}{Leonel}\\ 
\hline \\[-1.8ex] 
 $\texttt{retained}$   & 1.85^{***} & 2.37^{***} & 4.91^{***} & 3.10^{***} & 3.02^{***} & 4.59^{***} & 1.10^{*} & -.22 & 2.93^{***} \\ 
  & (.33) & (.33) & (.41) & (.43) & (.32) & (.44) & (.58) & (.75) & (.38) \\ 
  & & & & & & & & & \\ 
 $\texttt{lost}$       & 1.29^{*} &  &  & 1.27^{***} &  & 1.46^{*} &  &  &  \\ 
  & (.68) &  &  & (.47) &  & (.81) &  &  &  \\ 
  & & & & & & & & & \\ 
 $\texttt{delivered}$  & .86^{***} & .76^{***} & 1.46^{***} & .51^{*} & .93^{***} & .26 & .51 & .85^{***} & .26 \\ 
  & (.25) & (.27) & (.34) & (.30) & (.27) & (.34) & (.37) & (.27) & (.33) \\ 
  & & & & & & & & & \\ 
 $\texttt{interested}$ & .35 & 1.03^{***} & 1.34^{***} & .82^{***} & .52^{**} & .74^{**} & .71^{**} & .28 & .57^{*} \\ 
  & (.24) & (.27) & (.34) & (.28) & (.26) & (.33) & (.36) & (.27) & (.31) \\ 
  & & & & & & & & & \\ 
 $\texttt{smartphone}$ & -.27 & .37 & -.18 & -.47^{*} & .21 & -.05 & -.43 & .26 & -.42 \\ 
  & (.24) & (.27) & (.31) & (.28) & (.26) & (.31) & (.35) & (.27) & (.30) \\ 
  & & & & & & & & & \\ 
 $\texttt{panista}$    & .15 & -.11 & -.03 & 1.18^{***} & .02 & .80^{*} & .78^{*} & .34 & 1.15^{***} \\ 
  & (.39) & (.41) & (.52) & (.35) & (.41) & (.44) & (.47) & (.39) & (.41) \\ 
  & & & & & & & & & \\ 
 $\texttt{priista}$    & .37 & .15 & -.01 & -.21 & .17 & .74^{**} & .43 & .19 & .16 \\ 
  & (.28) & (.30) & (.38) & (.37) & (.29) & (.35) & (.41) & (.31) & (.39) \\ 
  & & & & & & & & & \\ 
 $\texttt{morenista}$  & -.07 & .59 & .26 & .76 & -1.17 &  & -.26 & -1.01 & .88 \\ 
  & (.63) & (.51) & (.74) & (.55) & (1.04) &  & (1.05) & (1.03) & (.56) \\ 
  & & & & & & & & & \\ 
 Intercept             & -3.03^{***} & -3.82^{***} & -4.45^{***} & -3.48^{***} & -3.49^{***} & -3.99^{***} & -3.87^{***} & -3.29^{***} & -3.58^{***} \\ 
  & (.25) & (.30) & (.39) & (.30) & (.28) & (.35) & (.37) & (.28) & (.30) \\ 
  & & & & & & & & & \\ 
\hline \\[-1.8ex] 
Observations & \multicolumn{1}{c}{1,008} & \multicolumn{1}{c}{1,008} & \multicolumn{1}{c}{1,008} & \multicolumn{1}{c}{1,008} & \multicolumn{1}{c}{1,008} & \multicolumn{1}{c}{1,008} & \multicolumn{1}{c}{1,008} & \multicolumn{1}{c}{1,008} & \multicolumn{1}{c}{1,008} \\ 
Log Likelihood & \multicolumn{1}{c}{-262.32} & \multicolumn{1}{c}{-231.34} & \multicolumn{1}{c}{-169.84} & \multicolumn{1}{c}{-205.60} & \multicolumn{1}{c}{-235.20} & \multicolumn{1}{c}{-175.64} & \multicolumn{1}{c}{-147.10} & \multicolumn{1}{c}{-229.85} & \multicolumn{1}{c}{-182.89} \\ 
\hline 
\hline \\[-1.8ex] 
    \multicolumn{10}{r}{\footnotesize{$^{*}$p$<$.1; $^{**}$p$<$.05; $^{***}$p$<$.01}} \\ %[-1.8ex]
\end{tabular}
}
  \caption{Regression results. All models estimated with logit, standard errors in parentheses.} 
  \label{T:regs} 
\end{sidewaystable} 

\end{document}
