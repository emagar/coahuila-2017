Excerpts from popkin

``There is something rather miraculous about the fact that citizens believe that leaders selected by balloting are ... entitled to govern. Part of the reason for this belief is that campaigns are able to reach people and involve them in the election. It is worth remembering that the term /campaign/ is derived from the French word for `open country' and brings to politics some of its original military use: in a military campaign, an army left its barracks in the capital city for operations in the field, or open country, This is an apt metaphor for politics, because campaigns bring politicians out of the capital city into the open country, where they must engage their political opponents in a series of battles conducted in full view of their countrymen, who will judge each contest. To arouse public opinion and generate support for their cause, they must defend their old policies, sell new policies, and justify their rule.'' p. 8

``Campaigns reach most people through the media. Besides attracting attention to the campaign `horse race,' the media play a critical role in shaping voters' limited information about the world, their limited knowledge about the links between issues and offices, their limited understanding of the connections between public policy and the immediate consequences for themselves, and their views about what kind of person a president should be. The campaign and media, in other words, influence the voter's frame of reference, and can thereby change his or her vote.'' p. 9

``I propose to view the voter as an investor [not a consumer] and the vote as a reasoned investment in collective goods, made with costly and imperfect information under conditions of uncertainty. ... Many consumer decisions involve clear alternatives and immediate results, but a decision about voting always involves uncertainty and the prospects of a long-term payoff.'' p. 10

``In 1952, campaign buttons said `I like Ike,' but at rallies people said `/We/ like Ike.' ... The transformation of `What have you done for me lately?' into `What have you done for /us/ lately?' is the essence of campaigning. Transforming unstructured and diverse interests into a single coalition, making a single cleavage dominant, requires the creation of new constituencies and political identities. It requires the aggregation of countless /I/s into a few /we/s. Behind the /we/s, however, are people who are still reasoning about the ways in which their lives and government policies are related.'' p. 12

``Public choices also differ from some private choices because they involve the provision of services. A politician is promising to deliver a future product about which the voter may have limited understanding, so the vote involves uncertainty about whether the product can be delivered, and, if so, whether it will perform as promised. Thus the voter has to assess the politician's ability to accoplish what he or she promises. ... To deliver promised benefits, a politician must do more than attract enough votes; he or she must attract the support of other politicians as well. For this reason, voters consider not only the personal characteristics of their candidate, but also the other politicians with whom he or she is affiliated.'' p. 11



